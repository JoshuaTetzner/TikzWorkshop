% Tikz Trek: Navigating the Basics of LaTex Graphics

% This is a Demo file for an introduction to Tikz. 
% We can only introduce some of the basics of creating graphics and plots in Tikz.
% More details on the basics can be found here: 
% https://de.overleaf.com/learn/latex/TikZ_package
% https://de.overleaf.com/learn/latex/LaTeX_Graphics_using_TikZ%3A_A_Tutorial_for_Beginners_(Part_1)%E2%80%94Basic_Drawing

% The graphics in this talk follow the experiment of Nils Arbeiter (https://eln.elaine.uni-rostock.de/experiments.php?mode=view&id=5087).


\documentclass[journal]{IEEEtran}

\usepackage{tikz}
\usepackage{graphicx}
\usepackage{pgfplots}
\usetikzlibrary{shapes.geometric}
\usepgfplotslibrary{groupplots}
\usepackage{lipsum}
\usepackage{siunitx}

\usepackage{figures/settings}

\begin{document}

\title{Bare Demo of IEEEtran.cls\\ for IEEE Journals}

\author{Michael~Shell,~\IEEEmembership{Member,~IEEE,}
        John~Doe,~\IEEEmembership{Fellow,~OSA,}
        and~Jane~Doe,~\IEEEmembership{Life~Fellow,~IEEE}% <-this % stops a space
\thanks{M. Shell was with the Department
of Electrical and Computer Engineering, Georgia Institute of Technology, Atlanta,
GA, 30332 USA e-mail: (see http://www.michaelshell.org/contact.html).}% <-this % stops a space
\thanks{J. Doe and J. Doe are with Anonymous University.}% <-this % stops a space
\thanks{Manuscript received April 19, 2005; revised August 26, 2015.}}

\markboth{Journal of \LaTeX\ Class Files,~Vol.~14, No.~8, August~2015}%
{Shell \MakeLowercase{\textit{et al.}}: Bare Demo of IEEEtran.cls for IEEE Journals}

\maketitle


\begin{abstract}
\lipsum[1-1]
\end{abstract}

\begin{IEEEkeywords}
IEEE, IEEEtran, journal, \LaTeX, paper, template.
\end{IEEEkeywords}




\IEEEpeerreviewmaketitle



\section{Scetching}
\IEEEPARstart{T}{his} is a  demo file.
\lipsum[1-2]


\begin{figure}[t]
    \centering
    %
\begin{tikzpicture}
    %TBD
\end{tikzpicture}
    \caption{This is my first scetch of a nice experiment provided by Nils.}
    \label{fig:labelA}
\end{figure}



\lipsum[1-2]
\subsection{Subsection Heading Here}
Subsection text here.
\begin{figure}[t]
    \centering
    %\begin{tikzpicture}
    \tikzstyle{every node}=[font=\footnotesize]
	
\end{tikzpicture}
    \caption{Caption}
    \label{fig:labelB}
\end{figure}
\lipsum[1-2]
\begin{figure}[t]
    \centering
    %\begin{tikzpicture}
    \tikzstyle{every node}=[font=\footnotesize]
    %TBD
\end{tikzpicture}
    \caption{Caption}
    \label{fig:labelC}
\end{figure}
\lipsum[1-2]
\begin{figure}[t]
    \centering
    %\begin{tikzpicture}
    \tikzstyle{every node}=[font=\footnotesize]
	
\end{tikzpicture}
    \caption{Caption}
    \label{fig:labelD}
\end{figure}
% needed in second column of first page if using \IEEEpubid
%\IEEEpubidadjcol

\subsubsection{Subsubsection Heading Here}
Subsubsection text here.

\section{Conclusion}
The conclusion goes here.



\begin{thebibliography}{1}

\bibitem{IEEEhowto:kopka}
H.~Kopka and P.~W. Daly, \emph{A Guide to \LaTeX}, 3rd~ed.\hskip 1em plus
  0.5em minus 0.4em\relax Harlow, England: Addison-Wesley, 1999.

\end{thebibliography}

\end{document}


